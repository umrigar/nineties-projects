\batchmode
\documentstyle[twocolumn]{article}
\makeatletter

\pagestyle{empty}
\footskip 2in 

\hoffset= -0.3in
\voffset= -1.0in
\setlength{\textwidth}{6.5in}\setlength{\textheight}{9in}
\setlength{\oddsidemargin}{0.3in}\setlength{\evensidemargin}{0.3in}
\setlength{\parindent}{0pt}\setlength{\parskip}{1.5ex}







\title{Fully Static Dimensional Analysis with C\raise 0.2ex \hbox{++}\ }
\author{
Zerksis D. Umrigar\\ 
447 Plaza Dr., 5-57, Vestal NY 13850\\ 
(607) 729-5919\\ 
Email: umrigar@cs.binghamton.edu
}
\date{}

\newcommand {\cpp}{C\raise 0.2ex \hbox{++}}

\newcommand {\gpp}{G\raise 0.2ex \hbox{++}}

\newcommand {\meter}{\hbox{Meter}}

\newcommand {\kg}{\hbox{Kg}}

\newcommand {\second}{\hbox{Sec}}

\makeatother
\newenvironment{tex2html_wrap}{}{}
\newwrite\lthtmlwrite
\def\lthtmltypeout#1{{\let\protect\string\immediate\write\lthtmlwrite{#1}}}%
\newbox\sizebox
\begin{document}
\pagestyle{empty}
\stepcounter{section}
\stepcounter{section}
\stepcounter{section}
\stepcounter{section}
{\newpage
\clearpage
\samepage \begin{figure}{\footnotesize
\begin{verbatim}01 #include <iostream.h>
02 
03 #include <math.h>
04 
05 #define N_DIM 3
06 #define DIM_VAL_TYPE double
07 //#define DIM_NO_CHECK
08 
09 #include "Dim.h"
10 
11 //Name each dimensional axis.
12 #define MASS_DIM     0
13 #define LENGTH_DIM   1
14 #define TIME_DIM     2
15 
16 //Declare units.
17 UNIT(Gm, MASS_DIM, 1.0, "gm");
18 UNIT(Cm, LENGTH_DIM, 1.0, "cm");
19 UNIT(Feet, LENGTH_DIM, 2.54*12, "feet");
20 UNIT(Sec, TIME_DIM, 1.0, "sec");
21 
22 typedef DIM( 1,  0,  0) Mass;
23 typedef DIM( 0,  1,  0) Length;
24 typedef DIM( 0,  0,  1) Time;
25 typedef DIM( 0,  1, -2) Acceleration;
26 typedef DIM( 1,  0, -1) Friction;
27 typedef DIM( 0,  1, -1) Velocity;
28 typedef DIM( 0,  0,  0) Number;\end{verbatim}
}


\label{Fig:Dec}
\end{figure}
}

{\newpage
\clearpage
\samepage \setbox\sizebox=\hbox{$\ldots$}\lthtmltypeout{latex2htmlSize :tex2html_wrap_inline290: \the\ht\sizebox::\the\dp\sizebox.}\box\sizebox
}

{\newpage
\clearpage
\samepage \begin{figure}{\footnotesize
\begin{verbatim}01 static Length height(Length x) {
02   return 0.5 * x;
03 }
04 
05 int main() {
06   const Acceleration g= 980.7 * Cm/(Sec*Sec);
07   const Mass blockMass= 1000 * Gm;
08   const Length deltaX= 0.01 * Cm;
09   const Friction friction= 20 * Gm/Sec;
10   const Number a= 1.0;
11   Length x, y;       //Initialized to 0.
12   const Time deltaT= 0.1*Sec;
13   Velocity v;
14   Time maxTime;
15 
16   cout << "Enter initial velocity (cm/sec): " 
17        << flush;
18   cin >> Cm/Sec >> v;
19   cout << "Enter time limit (sec): " << flush;
20   cin >> Sec >> maxTime;
21 
22   for (Time t= 0*Sec; t < maxTime; t+= deltaT) {
23     Number slope= (height(x + deltaX) - y)/deltaX;
24     Number cosAngle= 
25       1.0/sqrt(1 + +(slope * slope));
26     x+= v * deltaT * cosAngle;
27     y= height(x);
28     v+= deltaT * (g * slope * cosAngle - 
29                   v * friction / blockMass);
30     cout << "At T= " << Sec_ << t << " ";
31     cout << "X= " << Feet_ << x << "; ";
32     cout << " Y= " << Feet_ << y << "; ";
33     cout << " and V= " << Feet_/Sec_ << v << "; ";
34     cout << '\n';
35   }
36 
37   return 0;
38 }\end{verbatim}
}


\label{Fig:Code}
\end{figure}
}

\stepcounter{section}
\stepcounter{section}
\stepcounter{section}

\end{document}
